\documentclass[aspectratio=169,xcolor=dvipsnames]{beamer}
\usepackage[T2A]{fontenc}
\usepackage[utf8]{inputenc}
\usepackage[english,russian]{babel}
\usepackage{epsfig}
\usepackage{verbatim}
\usepackage{gnuplottex}
\usepackage{listings}
\usepackage{hyperref}
\usepackage{graphicx}
\usepackage{booktabs}
 

\usetheme{Simple}
%\setbeameroption{show notes}

\subtitle[короткое название]{Петрозаводский государственный университет
Институт математики и информационных технологий
Кафедра Информатики и математического обеспечения}

\author[Автор]{Бровкин Артём Павлович 22207}
\institute[ПетрГУ]
{
\Large{\textcolor{blue}{Создание чат-бота на платформе Discord}} \\\vspace{20px}
\small{Направление подготовки бакалавриата
09.03.04 Программная инженерия
Профиль направления подготовки бакалавриата
“Системное и прикладное программное обеспечение”} \vspace{20px}

\normalsize{Научный руководитель: к.т.н., доцент кафедры ИМО, C. А. Марченков}

    \vskip 3pt
}
\date{} 

\begin{document}

\begin{frame}
    \titlepage
\end{frame}
   
\begin{frame}{Цели}

\begin{enumerate}
    \item приобрести навыки и опыт работы с чат-ботом на платформе Discord;
    \item повысить свою квалификацию по ходу работы с JavaScript, Node.js, JSON;
    \item закрепить имеющиеся навыки во время работы с:
    \begin{itemize}
     \item языками разметки: LaTeX, Beamer, HTML.\vspace{3px}
    \item языком программирования: JavaScript.\vspace{3px}
    \item языком таблиц стиле: CSS.\vspace{3px}
    \item веб-сервисом для хостинга: GitHub.\vspace{3px}
    \end{itemize}
\end{enumerate}
\end{frame}

\begin{frame}{Задачи}

\begin{itemize}
    \item создание дискорд сервера для тестирования работы бота;
    \item создание бота;
    \item активация бота и его тестирование ;
    \item добавление большего функционала (добавляем команды);
    \item создание сайта для скачивания бота другим пользователям;
    \item проанализировать полученные результаты и сформулировать выводы, что удалось реализовать, что неудалось, какой получили опыт в результате работы;
    \item создание документации;
    \item отправление документации и кода на репозиторий GitHub;
\end{itemize}
\end{frame}

\begin{frame}{Актуальность|платформа|боты}
\begin{figure}
\includegraphics<1->[width=0.7\linewidth]{pictures/diskkk.png}
\end{figure}
\end{frame}

\begin{frame}{Создание дискорд сервера}
\begin{figure}
\includegraphics<1->[width=0.32\linewidth]{pictures/server.png}
\includegraphics<1->[width=0.4\linewidth]{pictures/server2.png}
\end{figure}
\end{frame}

\begin{frame}{Создание бота}
\begin{figure}
\includegraphics<1->[width=0.9\linewidth]{pictures/bot2.png}
\end{figure}
\end{frame}

\begin{frame}{Создание бота}
\begin{figure}
\includegraphics<1->[width=0.9\linewidth]{pictures/bot3.png}
\end{figure}
\end{frame}

\begin{frame}{Создание бота}
\begin{figure}
\includegraphics<1->[width=0.9\linewidth]{pictures/bot4.png}
\end{figure}
\end{frame}

\begin{frame}{Активируем и тестируем бота}
\begin{figure}
\includegraphics<1->[width=0.4\linewidth]{pictures/node.png}
\includegraphics<1->[width=0.4\linewidth]{pictures/vsc.png}
\end{figure}
\end{frame}

\begin{frame}{Активируем и тестируем бота}
\begin{center}
Устанавливаем пакеты
\end{center}
npm init \\
npm install \\
npm install discord.js axios dotenv
\begin{figure}
\includegraphics<1->[width=0.4\linewidth]{pictures/papki.png}
\end{figure}
\end{frame}


\begin{frame}{Активируем и тестируем бота}
\begin{figure}
\end{figure}
    \begin{figure}
    \centering
    \begin{minipage}[b]{0.4\textwidth}
    \includegraphics<1->[width=1.4\linewidth]{pictures/diskjs.png}
    \end{minipage}
    \hfill
    \begin{minipage}[b]{0.4\textwidth}
    \includegraphics<1->[width=1.7\linewidth]{pictures/js.png}
    \end{minipage}
    \end{figure}
\end{frame}

\begin{frame}{Активируем и тестируем бота}
\begin{figure}
\includegraphics<1->[width=0.7\linewidth]{pictures/zap.png}\vskip 12pt
\includegraphics<1->[width=0.6\linewidth]{pictures/test.png}
\end{figure}
\end{frame}

\begin{frame}{Добавляем больше команд}
\begin{figure}
\includegraphics<1->[width=0.75\linewidth]{pictures/help.png}
\end{figure}
\end{frame}

\begin{frame}{Добавляем больше команд}
\begin{figure}
\end{figure}
    \begin{figure}
    \centering
    \begin{minipage}[b]{0.4\textwidth}
    \includegraphics<1->[width=1.48\linewidth]{pictures/clearkod.png}
    \end{minipage}
    \hfill
    \begin{minipage}[b]{0.4\textwidth}
    \includegraphics<1->[width=1.5\linewidth]{pictures/clear2.png}
    \end{minipage}
    \end{figure}
\end{frame}

\begin{frame}{Создаём сайт}
\begin{figure}
\includegraphics<1->[scale=0.25]{pictures/site.png}
\includegraphics<1->[width=0.45\linewidth]{pictures/sss.png}
\end{figure}
\end{frame}

\begin{frame}{Заключение}
строк кода: 265 (не учитывая библиотеки и сайт)\vspace{10px}

\textbf{Выполненые задачи:} \vspace{5px}
\begin{itemize}
    \item создан дискорд сервер для тестирования работы бота;
    \item создан бот;
    \item бот активирован и протестирован; 
    \item расширен функционал и добавены новые команды;
    \item создан сайт для скачивания бота другим пользователям.
\end{itemize}
\end{frame}

\begin{frame}{Спасибо за внимание!!!}
\begin{figure}
\includegraphics<1->[width=1\linewidth]{pictures/phon.png}
\end{figure}
\end{frame}

\end{document}