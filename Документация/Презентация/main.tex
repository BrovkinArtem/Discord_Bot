\documentclass[aspectratio=169,xcolor=dvipsnames]{beamer}
\usepackage[T2A]{fontenc}
\usepackage[utf8]{inputenc}
\usepackage[english,russian]{babel}
\usepackage{epsfig}
\usepackage{verbatim}
\usepackage{gnuplottex}
\usepackage{listings}
\usepackage{hyperref}
\usepackage{graphicx}
\usepackage{booktabs}
 

\usetheme{Simple}
%\setbeameroption{show notes}

\subtitle[короткое название]{Петрозаводский государственный университет
Институт математики и информационных технологий
Кафедра Информатики и математического обеспечения}

\author[Автор]{Бровкин Артём Павлович 22207}
\institute[ПетрГУ]
{
\Large{\textcolor{blue}{Создание бота на платформе Discord}} \\\vspace{20px}
\small{Направление подготовки бакалавриата
09.03.04 Программная инженерия
Профиль направления подготовки бакалавриата
“Системное и прикладное программное обеспечение”} \vspace{20px}

\normalsize{Научный руководитель: к.т.н., доцент кафедры ИМО, C. А. Марченков}

    \vskip 3pt
}
\date{} 

\begin{document}

\begin{frame}
    \titlepage
\end{frame}

\begin{frame}{Что за платформа и кто такие боты}
\begin{figure}
\includegraphics<1->[width=0.7\linewidth]{pictures/diskkk.png}
\end{figure}
\end{frame}

\begin{frame}{Создание дискорд сервера}
\begin{figure}
\includegraphics<1->[width=0.32\linewidth]{pictures/server.png}
\includegraphics<1->[width=0.4\linewidth]{pictures/server2.png}
\end{figure}
\end{frame}

\begin{frame}{Создание бота}
\begin{figure}
\includegraphics<1->[width=0.9\linewidth]{pictures/bot2.png}
\end{figure}
\end{frame}

\begin{frame}{Создание бота}
\begin{figure}
\includegraphics<1->[width=0.9\linewidth]{pictures/bot3.png}
\end{figure}
\end{frame}

\begin{frame}{Создание бота}
\begin{figure}
\includegraphics<1->[width=0.9\linewidth]{pictures/bot4.png}
\end{figure}
\end{frame}

\begin{frame}{Активируем и тестируем бота}
\begin{figure}
\includegraphics<1->[width=0.4\linewidth]{pictures/node.png}
\includegraphics<1->[width=0.4\linewidth]{pictures/vsc.png}
\end{figure}
\end{frame}

\begin{frame}{Активируем и тестируем бота}
\begin{center}
Устанавливаем пакеты
\end{center}
npm init \\
npm install \\
npm install discord.js axios dotenv
\begin{figure}
\includegraphics<1->[width=0.4\linewidth]{pictures/papki.png}
\end{figure}
\end{frame}

\begin{frame}{Активируем и тестируем бота}
\begin{figure}
\includegraphics<1->[width=0.7\linewidth]{pictures/zap.png}\vskip 12pt
\includegraphics<1->[width=0.6\linewidth]{pictures/test.png}
\end{figure}
\end{frame}

\begin{frame}{Добавляем больше команд}
\begin{figure}
\includegraphics<1->[width=0.75\linewidth]{pictures/help.png}
\end{figure}
\end{frame}

\begin{frame}{Добавляем больше команд}
\begin{figure}
\includegraphics<1->[width=0.35\linewidth]{pictures/math.png}
\includegraphics<1->[width=0.6\linewidth]{pictures/kurs.png}
\end{figure}
\end{frame}

\begin{frame}{Создаём сайт}
\begin{figure}
\includegraphics<1->[width=0.42\linewidth]{pictures/site.png}
\includegraphics<1->[width=0.35\linewidth]{pictures/site2.png}
\end{figure}
\end{frame}

\begin{frame}{Вывод}
\begin{figure}
\includegraphics<1->[width=0.8\linewidth]{pictures/ksid.png}
\end{figure}
\end{frame}

\begin{frame}{Спасибо за внимание!!!}
\begin{figure}
\includegraphics<1->[width=1\linewidth]{pictures/phon.png}
\end{figure}
\end{frame}

\end{document}